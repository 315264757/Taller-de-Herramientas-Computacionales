\documentclass[letterpaper, 12pt, oneside]{article} 
\usepackage{amsmath}
\usepackage{graphicx}
\usepackage{xcolor}
\graphicspath{{Documents/THC/Clases/Latex/Imagenes/}}

\title{\Huge{Taller de Herramientas Computacionales}}
\author{Hector Chaparro Reza}
\date{Enero 14, 2019}

\begin{document}
	\maketitle
	\it Este es la primer bit\'acora del curso intersemestral de Taller de Herramientas Computacionales, este comenz\'o la fecha (07/01/2019). Aqui resumir\'e lo m\'as relevante de la teor\'ia. Los problemas y las soluciones que surgieron a lo largo de la pr\'actca ser\'an comentadas en el cuerpo de la bit\'acora.
	\newpage
	
	\title{\Huge{Introduccion y comandos basicos para bash de linux}}
	
	\textbf{Sistemas operativos, distribuciones y contexto}
	
	\begin{enumerate}
		\item {Sistemas operativos Linux y Windows:}
		\begin{enumerate}
			\item Linux:\\
			El profesor Luis nos habl\'o de las ventajas de usar el sistema operativo Linux, primero y muy importante es totalmente gratis por ser un GPL (general public license), esto tambi\'en es ventajoso ya que cualquiera con los conocimientos adecuados puede contribuir con actualizaciones para que el software GNU/Linux se desarrolle de una manera que sea eficiente para las personas que lo usan.\\
			¿Pero qu\'e personas lo usan? en su mayor\'ia programadores gracias a la libertad que existe para manipular el software. Otra de las ventajas es que los requisitos m\'inimos para usar las actualizaciones m\'as recientes son un procesador de 400 Mhz, al menos 1 GB de RAM y  GB de espacio en el disco, esto no es mucho y muestra que Linux es eficiente. Una desventaja de Linux es que no tiene una cantidad tan extensa de software como otros OS.\\	
			\item Windows:\\
			Hicimos una comparaci\'on con Linux y la ventaja de que tiene Windows es que es m\'as amigable para usuarios que no tienen mucho conocimiento en computaci\'o gracias a la interfaz grafica que viene por default en el sistema, desgraciadamente esto l\'mita y dificulta la ineracci\'on con la CPU esto tambi\'en se debe a que hay que pagar para usarlo y sus licencias son de acceso privado, otro defecto es que sus requisitos minimos para el sofware m\'as actual es de 2 Ghz que es m\'as del doble que el de Linux, RAM de 1 GB para sistemas de 32bits y de 2 GB ara 64bits adem\'as de un espacio de 16 y 20 GB en el disco respectivamente. La ventaja de usar Windows est\'a en la amplia gama de software que existe para entretenimieto, media y juegos de video.
		\end{enumerate}
	
		\item Distribuciones:\\
		Las distribuciones de software Linux est\'an basadas en el Kernel Linux (el kernel es el nucleo que conecta servidores para usar el hardware y el software en conjunto), estos incluyen determinads paquetes de software en com\'un con licencias libres. Estas distribuciones difieren en el modo en el que se comunican con la CPU por medio de un inerprete y la estructura gr\'afica que manejan. Algunos ejemplos de distribuciones que vimos en clase son:
		\begin{enumerate}
			\item Fedora
			\item Ubuntu
			\item Kubuntu
			\item Debian
			\item OpenSuSE
		\end{enumerate}
	
		\item Comandos, interacci\'on con la terminal:\\
		Para concluir la clase hablamos del bash de Linux el cual es un programa que funciona como inerprete de instrucciones que los usuarios escribimos con el fin de poder realizar diferentes procesos y tareas, algunos de los comando de la termina que vimos en clase fueron:
		\begin{enumerate}
			\item ls	(listado del directorio y si agregamos -l muestra valores, owner,group y all, que determinan que se puede hacer con los archivos)
			\item touch	(crea un archivo temporal)
			\item chmod	(cambia los valores bits del archivo o,g,a)
			\item python	(describe version de python instalada y abre programa)
			\item echo (deja una linea) 
			\item set	(muestra variables de entorno)
			\item pwd	(donde estoy mostrando la ruta)
			\item cd
		\end{enumerate}
	\end{enumerate}\part{title}
\end{document}