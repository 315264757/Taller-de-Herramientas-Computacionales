\documentclass{beamer}
\usepackage{graphicx}
\graphicspath{/home/prime/Pictures}
\usepackage[utf8]{inputenc}
%\usepackage[spanish]{babel}
\usetheme{Berkeley}

%%%%%%%%%%%%%%%%%%%%%%%%%%%%%%%%%%%%%%%%%%%%%%%%%%%%%%
%Esto se utiliza para el tema Bergen                 %
%\def\insertauthorindicator{Quién}% Default "Quién"  %
%\def\insertinstituteindicator{UNAM}% Default "Dónde"%
%\def\insertdateindicator{Fecha}% Default "Cuándo?"  %
%\title{Taller de Herramientas computacionales}      %
%\author{Hector Chaparro Reza}              %
%\date{\today}                                       %
%%%%%%%%%%%%%%%%%%%%%%%%%%%%%%%%%%%%%%%%%%%%%%%%%%%%%%

\title{Taller de Herramientas Computacionales}
\author{Héctor Chaparro Reza}
\date{22/01/2019}

\begin{document}

\begin{frame}
\frametitle{Diapositiva1}
\begin{center}
	\includegraphics[scale=]{/home/prime/Pictures/1.png}
	Esta es la bit\'acora número trece del curso intersemestral de Taller de Herramientas Computacionales, con la fecha (18/01/2019). Aqui resumir\'e lo m\'as relevante de la teor\'ia. Los problemas y las soluciones que surgieron a lo largo de la pr\'actca ser\'an comentadas en el cuerpo de la bit\'acora.
\end{center}
\end{frame}

\begin{frame}[fragile]
	\begin{verbatim}
	G="""
	El punto en R3 es:
	(x,y,z)={laX:.2f},{laY:g},{laZ:G} 
	""".format(laX=x,laY=y,laZ=z)
	print G
	
	#Conversion de tipos
	print 'Algo ' + str(x)
	
	print float(str(x))*3
	#print 'x es de tipo %s' % type(x)._name_
	print type(5)
	print type('5')
	
	print "Coordenadas de la flecha\n"
	x=input("¿Cual es la abcisa?\n")
	y=input("¿Cual es la ordenada?\n")	
	\end{verbatim}
\end{frame}
	
\end{document}