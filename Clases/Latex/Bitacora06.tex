\documentclass[letterpaper, 12pt, oneside]{article} 
\usepackage{amsmath}
\usepackage{graphicx}
\usepackage{xcolor}
\graphicspath{{Documents/THC/Clases/Latex/Imagenes/}}

\title{\Huge{Taller de Herramientas Computacionales}}
\author{Hector Chaparro Reza}
\date{Enero 15, 2019}

\begin{document}
	\maketitle
	\it Esta es la sexta bit\'acora del curso intersemestral de Taller de Herramientas Computacionales, con la fecha (11/01/2019). Aqui resumir\'e lo m\'as relevante de la teor\'ia. Los problemas y las soluciones que surgieron a lo largo de la pr\'actca ser\'an comentadas en el cuerpo de la bit\'acora.\\
	\newpage
	
	
	\title{\Huge{Resolviendo problemas}}\\
	
	\textbf{Definir, analizar, delimitar y resolver un problema}\\
	
	\begin{enumerate}
		\item {Retomando el ejemplo de la pelota}
		\begin{enumerate}
			\item M\'odulos\\
			Anteriormente hab\'iamos programado un script que que calculaba la posici\'on de una pelota dada la velocidad inicial, el tiempo que tarda en caer al suelo y la fuerza de gravedad que la tierra ejerce sobre otros cuerpos. Pero que tal que queremos refinar nuestro programa para que al ejecutarlo en la terminal para que solo tengamos que ingresar los valores para que calcule cualquier posici'on en un tiro vertical. usamos el comando def seguido del nombre que utilizaremos para llamar al m\'odulo y despu\'es un parentesis en el cual ingresaremos los valores que ser\'an sustituidos en la ecuaci\'on que se nos de la gana (estos valores deben ir separados por comas). Al final necesitamos que la ecuaci\'on nos regrese el valor calculado y utilizamos el comando return()\\
			
			\item Imprime resultados:\\
			A la hora de correr el m\'odulo en la terminal de python habremos importado la ecuaci\'on y solo bastar\'a llamarlo y darle valores para que haga el c\'alculo pero para esto necesitamos que dentro del m\'odulo exista un print que especifique como es que nos va arrojar el resultado, con muchos decimales para una mayor precisi\'on o s\'olo un entero para tener una aproximaci\'on. para esto usamos el signo de porcentaje y las letras r,g,e minusculas y maysculas como se ve en el recuento del ejercicio que hicimos para poder identificar como imprime cada letra. \\
			
			
			
			-*- coding: utf-8 -*-\\
			v0 = 34\\
			g = 9.81\\
			t = 4.3\\
			y = v0*t - 1.0/2*g*t**2\\
			print 'La posicion de la pelota en el t=porcentajeE es porcentaje10.7fnporcentajef' porcentaje (t,y,t)\\
			def posicion(t,v0):\\
			y = v0*t - 1.0/2*g*t**2\\
			return(y)\\
			
			Al finalizar el ejercicio y refinal el programa de la pelota se qued\'o de tarea refinal el programa que hab\'iamos hecho con nuestro problema de f\'isica.
				
			Al final la ayudante Karla nos enseño los comandos b\'asicos para escribir en latex, coomo crear un documento, usar paquetes y practicamente todo lo que use para escribir esta bitacoras. Aun no puedo resolver el problema de las imagenes, yo supongo que estoy metiendo mal el path pero no se.
		\end{enumerate}
	\end{enumerate}
\end{document}