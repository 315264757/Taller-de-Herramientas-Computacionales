\documentclass[letterpaper, 12pt, oneside]{article} 
\usepackage{amsmath}
\usepackage{graphicx}
\usepackage{xcolor}
\graphicspath{{Documents/THC/Clases/Latex/Imagenes/}}
\usepackage{amssymb}

\title{\Huge{Taller de Herramientas Computacionales}}
\author{Hector Chaparro Reza}
\date{Enero 14, 2019}

\begin{document}
	\maketitle
	\it Esta es la sexta bit\'acora del curso intersemestral de Taller de Herramientas Computacionales, con la fecha (14/01/2019). Aqui resumir\'e lo m\'as relevante de la teor\'ia. Los problemas y las soluciones que surgieron a lo largo de la pr\'actca ser\'an comentadas en el cuerpo de la bit\'acora.\\
	\newpage
	
	
	\title{\Huge{Resolviendo problemas}}\\
	
	\textbf{Nuevos comandos. Definir, analizar, delimitar y resolver un problema}\\
	
	\begin{enumerate}
		\item {Problemas, problemas, problemas...}
		\begin{enumerate}
			\item M\'odulos\\
			Continuamos hablando de la manera en la que se solucionan los problemas y el profesor planteo el ejercicio de definir la raiz cuadrada con aproximaciones de nuevo y recalcó que era más facil primero hacer un analisis y escribir el problema y después codificarlo. Continuamos con algunos ejemplos de operaciones en el shell de python importando el paquete math y una leccion aqui fue que siempre que hagamos un import de una biblioteca y queramos usar una de sus funciones tenemos que usar la estructura "objeto.metodo(variable)". \\
			
			\item Condicional y ciclo:\\
			Llego la hora de empezar a afinar la herramienta que tenemos para resolver problemas en python, a la hora de hacer la tarea de la creación del módulo para calcular la posicion de una pelota logramos resolver el problema pero con limitaciones. Con el condicional if para hacer un módulo que nos permitiera aplicar el valor absoluto a cualquier nùmero y despues usamos el bucle ciclo while para definir correctamente el problema del cálculo de raices cuadradas por medio de aproximaciones las cuales dejamos con unos 5 o 6 decimales de precisión si no mal recuerdo con el porcentajeg en el print. A continuación dejare el codigo para que se pueda entender mejor lo que hicimos.\\
			
			
			
			\newpage
			\begin{verbatim}
			def vAbsoluto(x):
			if x>=0:
			return(x)
			else:
			return(-x)
			
			def raiz(x):
			h=x
			b=1.0
			e=0.0001
			while vAbsoluto(b-h)>e:
			h = (b+h)/2
			b = x/h
			return(b)
			
			print raiz(1.0)
			print raiz(4.0)
			print raiz(9.0)
			print raiz(9.1)
			print raiz(1000000.0)
			
			def raiz1(x):
			h=x
			b=1.0
			e=0.0001
			i=0 #cuenta el numero de veces que se ejecuta el ciclo
			while vAbsoluto(b-h)>e:
			i=i+1
			h = (b+h)/2
			b = x/h
			print "El ciclo se repitio %d veces"% (i)
			return(b)
			
			print raiz1(1.0)
			print raiz1(4.0)
			print raiz1(9.0)
			print raiz1(9.1)
			print raiz1(1000000.0)
			\end{verbatim}
			
			\newpage
			
			Al finalizar el ejercicio practicamos los condicionales con un ejercicio llamado la Diana que era separar por cuadrantes un plano y especificar que en ciertos lugares habría mas puntos y quedo de tarea mejorarlo.
				
			Al final la ayudante Karla nos enseño los comandos para escribir en latex, y pudo resolver el problema de las imagenes cambiandolas al home.
		\end{enumerate}
	\end{enumerate}
\end{document}