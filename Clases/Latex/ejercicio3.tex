\documentclass{book}
\usepackage[spanish]{babel}
\usepackage{graphicx}
\usepackage[utf8]{inputenc}
\usepackage{hyperref}
\usepackage{biblatex}


\usepackage{amsmath}
\usepackage{amssymb}
\title{Taller de Herramientas Computacionales}
\author{Héctor Chaparro Reza}
\date{17/Enero/2019}
\begin{document}
\maketitle
\tableofcontents
\section*{Introducción} Este ejercicio es una herramienta para redactar un documentoncon una estructura bien ordenada, con un arreglo más profesional en cuanto a la presentación.
\chapter{Uso Básico de Linux}
\section{Distribuciones de Linux}
\section{Comandos}


\chapter{Uso de Linux Básico}
\subsection*{Comandos}
Puedes buscar todos los comandos en google.
\url{www.google.com}\\
\hyperref[google]{www.google.com}
\chapter{Introducción a LaTeX}
\chapter{Introducción a Python}
\subsection*{Orientación a  Objetos}

\bibliography{Blah Blah BLah}
\begin{thebibliography}{9}
\bibitem{Libro}
Autor blah blah blah
\textit{Cualquier cosa}
Blablah blah, 2019.
\end{thebibliography}

\end{document}
