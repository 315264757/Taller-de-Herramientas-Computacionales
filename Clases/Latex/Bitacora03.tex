\documentclass[letterpaper, 12pt, oneside]{article} 
\usepackage{amsmath}
\usepackage{graphicx}
\usepackage{xcolor}
\graphicspath{{Documents/THC/Clases/Latex/Imagenes/}}

\title{\Huge{Taller de Herramientas Computacionales}}
\author{Hector Chaparro Reza}
\date{Enero 14, 2019}

\begin{document}
	\maketitle
	\it Este es la tercer bit\'acora del curso intersemestral de Taller de Herramientas Computacionales, con la fecha (09/01/2019). Aqui resumir\'e lo m\'as relevante de la teor\'ia. Los problemas y las soluciones que surgieron a lo largo de la pr\'actca ser\'an comentadas en el cuerpo de la bit\'acora. 
	\newpage
	
	\title{\Huge{Introducci\'on a Python}}\\
	
	\textbf{Conceptos b\'asicos, operaciones}\\
	
	\begin{enumerate}
		\item {Como usar Python:}
		\begin{enumerate}
			\item Instalaci\'on\\
			Por suerte los que usamos la distribuci\'on Fedora de Linux ya tiene instalado Python 2.7.15 por lo que directamente usamos el comando idle que abre la terminal de python en el cual podemos ejecutar comandos. Ah\'i mismo se puede abrir la opci\'on files y new file pero es m\'as r\'apido usar el comando ctrl+n. Esto abre un editor de texto donde podremos trabajar\\
			
			\item Operaciones:\\
			En la terminal de python aprend\'imos a escribir las operaciones matem\'aticas b\'asicas adici\'on(a+b) y su inversa(a-b), multiplicaci\'on(a*b) y su inversa(a/b) y potencia(a**b). Por un buen rato estuvimos probando en el shell de python diferentes operaciones y vimos que los resultados se redondeaban si no manejabamos puntos flotantes para mayor precisici\'on por ejemplo si hac\'iamos 1/2 esto nos daba 0 pero si divid\'iamos 1.0/2.0 nos daba el valor correcto o sea 0.5. Antes de seguir trabajando tuvimos una catedra de la importancia de analizar un problema para delimitarlo y buscar una soluci\'on. Finalmente qued\'o de tarea investigar un problema sobre f\'isica\\
			
		\end{enumerate}
	\end{enumerate}
\end{document}