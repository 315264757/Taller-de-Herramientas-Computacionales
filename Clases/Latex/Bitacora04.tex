\documentclass[letterpaper, 12pt, oneside]{article} 
\usepackage{amsmath}
\usepackage{graphicx}
\usepackage{xcolor}
\graphicspath{{Documents/THC/Clases/Latex/Imagenes/}}

\title{\Huge{Taller de Herramientas Computacionales}}
\author{Hector Chaparro Reza}
\date{Enero 14, 2019}

\begin{document}
	\maketitle
	\it Este es la cuarta bit\'acora del curso intersemestral de Taller de Herramientas Computacionales, con la fecha (10/01/2019). Aqui resumir\'e lo m\'as relevante de la teor\'ia. Los problemas y las soluciones que surgieron a lo largo de la pr\'actca ser\'an comentadas en el cuerpo de la bit\'acora. 
	\newpage
	
	\title{\Huge{Introducci\'on a Python}}\\
	
	\textbf{Conceptos b\'asicos, crear scripts, operaciones}\\
	
	\begin{enumerate}
		\item {Como usar Python:}
		\begin{enumerate}
			\item Instalaci\'on\\
			Al inicio continuamos con la discusi\'on sobre el an\'alisis de un problema y Luis menciono un principio clave para ayudarnos a programar que es el "divide y vencer\'as" el cual refiere a la parte de delimitar el problema y que pequeñas tareas resuelvan un problema mayor, hablamos sobre varios ejemplos en los que el profesor expon\'ia un problema aparentemente complejo pero al analizarlo con detenimiento, resultaba ser un principio muy b\'asico. El ejemplo que dio fue\\
			
			\item Scripts/programas:\\
			Ya en el editor de texto comenzamos a crear un script, o sea un programa que sea capaz de realziar una tarea o resovler un problema.  Posteriormente utilizamos un ejemplo de f\'sica en un problema de tiro vertical el cual se toma en cuenta diferentes variables como lo son la velocidad inicial, el tiempo y la fueza de gravedad de la tierra. tomando en cuenta lo aprendido el dia anterior hicimos el c\'alculo con flotantes, enteros, combinacion de ambos para ver que arrojar\'an resultados diferentes. Al terminar el script se vi\'o asi:\\
			
			-*+ coding: utf-8 -*-\\
			Hector Chaparro 315264757\\
			print 34*3 - 1/2 * 9.81**2\\
			print 34*3 - 1.0/2 * 9.81*3**2\\
			print 34*1 - 1.0/2 * 9.81*1**2\\
			print 34*1.5 - 1.0/2 * 9.81*1.5**2\\
			print 34*5 - 1.0/2 * 9.81*5**2\\
			v0 = 34\\
			g = 9.81\\
			t = 5\\
			y = v0*t - 1.0/2*g*t**2\\
			print y\\
			
			Y para terminar el script lo guardamos y lo ejecutamos en el shell de python con F5.
			
			Al final con el 
		\end{enumerate}
	\end{enumerate}
\end{document}