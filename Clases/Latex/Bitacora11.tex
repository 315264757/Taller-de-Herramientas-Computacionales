\documentclass[letterpaper, 12pt, oneside]{article} 
\usepackage{amsmath}
\usepackage{graphicx}
\usepackage{xcolor}
\graphicspath{{Documents/THC/Clases/Latex/Imagenes/}}
\usepackage{amssymb}

\title{\Huge{Taller de Herramientas Computacionales}}
\author{Hector Chaparro Reza}
\date{Enero 18, 2019}

\begin{document}
	\maketitle
	\it Esta es la bit\'acora número once del curso intersemestral de Taller de Herramientas Computacionales, con la fecha (18/01/2019). Aqui resumir\'e lo m\'as relevante de la teor\'ia. Los problemas y las soluciones que surgieron a lo largo de la pr\'actca ser\'an comentadas en el cuerpo de la bit\'acora.\\
	\newpage
	
	
	\title{\Huge{Ciclos for para listas}}\\
	
	\begin{enumerate}
		\item {Ejemplo11.py}
		\begin{enumerate}
			\item En el ejercicio usamos inputs para guardar valores en la terminal con el ciclo for, el cual es una iteración y para cada valor que generaba el ciclo con un append lo fuimos guradando en una lista creada anteriormente. También vimos ejemplos de como usar el rango, y yo lo entendi de manera que el range es el alcance del valor que ingresas por ejemplo si yo uso range de una lista, este va a ser el numero de elementos de la lista que le de a a evaluar:\\
			
			\begin{verbatim}
			# !/usr/bin/python
			# -*- coding: utf-8 -*-
			
			n = input("¿cuantos valores?")
			L=[]
			for i in range(n):
				valor = input("Dame el valor")
				L.append(valor)
			
			M=range(n)
			for i in M:
				valor = input("Dame el valor")
				M[i](valor)
			
			N=range(n)
			j=0
			while j<n:
				valor = input("Dame el valor")
				N[i](valor)
			
			L=range
			print "La lista es de tamano %d" % (len(L))
			print L
			
			OtraLista = input("")
			\end{verbatim}
			
			Estuvimos un rato jugando con listas metiendo y sacando datos con el ciclo for, uno que me ha sido muy util es utilizar en conjunto eun for con range y dentro de ese meter el len de la lista que me va a representar el tamaño de la lista. Después de esto utilizamos el codigo de conversión de grados centigrados a farenheit para crear un script que guardara valores en listas y en tuplas por medio del ciclo for.
			
			\begin{verbatim}
			def listaC(Cmin, Cmax, n):
				gradosC = []
				dC = (Cmax - Cmin)/float(n-1)
				# for(i=0;i<n;i++)
				for i in range(n):
					C = Cmin + i*dC
					gradosC.append(C)
					return gradosC
			
			def listaF(gradosC):
				gradosF = []
				for C in gradosC:
					F = (9.0/5)* C + 32
					gradosF.append(F)
					return gradosF
			
			def mostrarListas(gradosC, gradosF):
			#se podria usar n=(len(grados)): y en la linea de abajo usar n pero no es tan eficiente
				for i in range(len(gradosC)):
					C = gradosC[i]
					F = gradosF[i]
					print "%5.1f %5.1f" % (C, F)
			
			def mostrarListas1(gradosC, gradosF):
				for C, F in zip(gradosC, gradosF):
				print '%5d %5.1f' % (C,F)		
			\end{verbatim}
			
		
		\end{enumerate}
	\end{enumerate}
\end{document}