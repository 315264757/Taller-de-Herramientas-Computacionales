\documentclass{beamer}
\usepackage{graphicx}
\graphicspath{Pictures/}
\usepackage[utf8]{inputenc}
%\usepackage[spanish]{babel}
\usetheme{Berkeley}

\title{Taller de Herramientas Computacionales}
\author{Héctor Chaparro Reza}
\date{22/01/2019}

\begin{document}
	\begin{frame}
	\frametitle{Presentación}
	%\includegraphics[scale=]{Pictures/1.png}
\end{frame}

\begin{frame}[fragile]
	\begin{verbatim}
	G="""
	El punto en R3 es:
	(x,y,z)={laX:.2f},{laY:g},{laZ:G} 
	""".format(laX=x,laY=y,laZ=z)
	print G
	
	#Conversion de tipos
	print 'Algo ' + str(x)
	
	print float(str(x))*3
	#print 'x es de tipo %s' % type(x)._name_
	print type(5)
	print type('5')
	
	print "Coordenadas de la flecha\n"
	x=input("¿Cual es la abcisa?\n")
	y=input("¿Cual es la ordenada?\n")	
	\end{verbatim}
\end{frame}
	
\end{document}