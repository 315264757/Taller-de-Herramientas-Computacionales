\documentclass[letterpaper, 12pt, oneside]{article} 
\usepackage{amsmath}
\usepackage{graphicx}
\usepackage{xcolor}
\graphicspath{{Documents/THC/Clases/Latex/Imagenes/}}
\usepackage{amssymb}

\title{\Huge{Taller de Herramientas Computacionales}}
\author{Hector Chaparro Reza}
\date{Enero 18, 2019}

\begin{document}
	\maketitle
	\it Esta es la bit\'acora número doce del curso intersemestral de Taller de Herramientas Computacionales, con la fecha (18/01/2019). Aqui resumir\'e lo m\'as relevante de la teor\'ia. Los problemas y las soluciones que surgieron a lo largo de la pr\'actca ser\'an comentadas en el cuerpo de la bit\'acora.\\
	\newpage
	
	
	\title{\Huge{Presentaciones en latex y formato de listas en python}}\\
	
	\begin{enumerate}
		\item {Dando formato y presentaciones}
		\begin{enumerate}
			\item En está clase usamos el codigo de grados de la clase anerior para revisar de que manera el shell de python mostraba los datos, elementos de las listas y las listas en si. Primero vimos que al usar un for sin especificar que los elementos los queremos dentro de una lista, las listas de listas o matrices aparecerán no como listas si no como tuplas y a estas no se le pueden hacer modificaciones. También importamos de pprint una función llamada pprint para que las listas se mostrarán en forma vertical.
			
			\item Gran parte de la clase la dedicamos a aprender como hacer presentaciones en latex, usar el documento de clase beamer. Crear diapostitivas y darles un formato\\
			
			\begin{verbatim}
			\frametitle{Diapositiva1}
			\begin{center}
			\includegraphics[scale=]{/home/prime/Pictures/1.png}
			\end{center}
			\end{frame}
			
			\begin{frame}[fragile]
			\begin{verbatim}
			G="""
			El punto en R3 es:
			(x,y,z)={laX:.2f},{laY:g},{laZ:G} 
			""".format(laX=x,laY=y,laZ=z)
			print G
			
			#Conversion de tipos
			print 'Algo ' + str(x)
			
			print float(str(x))*3
			#print 'x es de tipo %s' % type(x)._name_
			print type(5)
			print type('5')
			
			print "Coordenadas de la flecha\n"
			x=input("¿Cual es la abcisa?\n")
			y=input("¿Cual es la ordenada?\n")	
		
			\end{verbatim}
			
		
		\end{enumerate}
	\end{enumerate}
\end{document}